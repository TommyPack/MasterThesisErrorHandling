\chapter{Business Process Model and Notation}
\label{cha:BPMN}
This chapter focuses on the Business Process Model and Notation (\gls{bpmn}) process modulation standard. First a short overview of the modell will be given. Afterwards the symbols will be described and the advantages of using such a model will be pointed out.\\
Advantage of using \gls{bpmn}

\section{Overview}
\label{sec:BPMNOverview}


Business Process Model and Notation (\gls{bpmn}) is a standard for business process
modeling that provides graphical notation for specifying business processes in a
Business Process Diagram (BPD),2 based on traditional flowcharting techniques.
The objective of \gls{bpmn} is to support business process modeling for both technical
users and business users, by providing notation that is intuitive to business users, yet
able to represent complex process semantics. The \gls{bpmn} 2.0 specification also provides execution semantics as well as mapping between the graphics of the notation
and other execution languages, particularly Business Process Execution Language
(BPEL).3
\gls{bpmn} is designed to be readily understandable by all business stakeholders.
These include the business analysts who create and refine the processes, the technical developers responsible for implementing them, and the business managers who
monitor and manage them. Consequently, \gls{bpmn} serves as a common language,
bridging the communication gap that frequently occurs between business process
design and implementation \cite{Rosing.2015b}.

\section{\gls{bpmn} Symbols}
\label{sec:BPMNSymbols}

\section{Advantage of using \gls{bpmn}}
\label{sec:BPMNGoal}